\newcommand{\pluginName}{Random Sources Support}
\newcommand{\pluginVersion}{1.0}

\input{../../DocumentationTemplate/Template}

\begin{document}

\PluginTitle{\pluginName}{\pluginVersion}

\section{Introduction}

The Random Source Support plug-in allows Fairmat to use external random stream sequences as input to the Monte Carlo simulation simulator.

This plug-in provides a general infrastructure and a file system based caching support for the random streams which allows developers to easily define new random sources by using the Fairmat extensions framework (See Section~\ref{sec.dev} for developers oriented information). 

\section{How to use the plug-in}

With random streams we indicate sequences of uniform distributed random numbers in the interval [0,1). Those numbers are then transformed to obtain any probability distribution.

The plug-in allows Fairmat to use different sequences of uniform random number generated by external sources. 
By default \emph{Random Source Support} allows you to select a binary raw file\footnote{The raw binary file is a sequence of double precision 8-bytes numbers.} or a text files containing the random numbers which will be used as source of randomness for all the subsequent Monte Carlo valuations.

In order choose a random file as input for the randomness you should do the following steps:

\section{How to add a new random source}
\label{sec.dev}
This section is dedicated to developers who wants  to define a new source of 
randomness.

In order to implement a new random source, you must write a
class which implement the interface \textbf{DVPLI.IRandomSource} and the extension node "/RandomSourcesSupport/RandomSource". 

In your implementation you must define how to generate the next random number and  the Random Sources Support plug-ins will account for the rest of the work.


\bibliographystyle{unsrt}
\bibliography{../../DocumentationTemplate/bibliography}

\end{document}
